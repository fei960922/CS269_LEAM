\documentclass{article}

% if you need to pass options to natbib, use, e.g.:
% \PassOptionsToPackage{numbers, compress}{natbib}
% before loading nips_2017
%
% to avoid loading the natbib package, add option nonatbib:
% \usepackage[nonatbib]{nips_2017}

\usepackage[final]{nips_2017}

% to compile a camera-ready version, add the [final] option, e.g.:
% \usepackage[final]{nips_2017}

\usepackage[utf8]{inputenc} % allow utf-8 input
\usepackage[T1]{fontenc}    % use 8-bit T1 fonts
\usepackage{hyperref}       % hyperlinks
\usepackage{url}            % simple URL typesetting
\usepackage{booktabs}       % professional-quality tables
\usepackage{amsfonts}       % blackboard math symbols
\usepackage{nicefrac}       % compact symbols for 1/2, etc.
\usepackage{microtype}      % microtypography
\usepackage{graphicx}
\usepackage{float}
\usepackage{placeins}
\usepackage{filecontents}

\begin{filecontents}{bbib.bib}
  @article{wang2018joint,
  title={Joint embedding of words and labels for text classification},
  author={Wang, Guoyin and Li, Chunyuan and Wang, Wenlin and Zhang, Yizhe and Shen, Dinghan and Zhang, Xinyuan and Henao, Ricardo and Carin, Lawrence},
  journal={arXiv preprint arXiv:1805.04174},
  year={2018}
}
@article{you2018attentionxml,
title={AttentionXML: Extreme Multi-Label Text Classification with Multi-Label Attention Based Recurrent Neural Networks},
author={You, Ronghui and Dai, Suyang and Zhang, Zihan and Mamitsuka, Hiroshi and Zhu, Shanfeng},
journal={arXiv preprint arXiv:1811.01727},
year={2018}
}
@misc{
jiao2019probabilistic,
title={Probabilistic Semantic Embedding},
author={Yue Jiao and Jonathon Hare and Adam Prügel-Bennett},
year={2019},
url={https://openreview.net/forum?id=r1xwqjRcY7},
}
@article{bahuleyan2017variational,
  title={Variational attention for sequence-to-sequence models},
  author={Bahuleyan, Hareesh and Mou, Lili and Vechtomova, Olga and Poupart, Pascal},
  journal={arXiv preprint arXiv:1712.08207},
  year={2017}
}
@inproceedings{deng2018latent,
  title={Latent alignment and variational attention},
  author={Deng, Yuntian and Kim, Yoon and Chiu, Justin and Guo, Demi and Rush, Alexander},
  booktitle={Advances in Neural Information Processing Systems},
  pages={9712--9724},
  year={2018}
}
\end{filecontents}


\title{Proposal : Research on Text classification}

% The \author macro works with any number of authors. There are two
% commands used to separate the names and addresses of multiple
% authors: \And and \AND.
%
% Using \And between authors leaves it to LaTeX to determine where to
% break the lines. Using \AND forces a line break at that point. So,
% if LaTeX puts 3 of 4 authors names on the first line, and the last
% on the second line, try using \AND instead of \And before the third
% author name.

\author{
  Yifei Xu (304880196), Yaxuan Zhu (704947072), Ruiqi Gao (104864885) \\
  Department of Statistics \\
  University of California, Los Angeles \\
  \texttt{fei960922@ucla.edu, yaxuanzhu@ucla.edu, ruiqigao@ucla.edu} \\
}

\begin{document}

\maketitle

\section{Introduction}

Text classification is a common task in natural language processing (NLP). It predict the most relevant label(s) to a given sentenses or paragraphs. It is widely used in tag recommendation, information retrieval, document categorization, etc. 

A good representation of text is key to this problem. There are multiple work on this field including \cite{you2018attentionxml} \cite{jiao2019probabilistic} \cite{bahuleyan2017variational} \cite{deng2018latent}. It is intuitively to introduce attention into text representation. Recently, LEAM \cite{wang2018joint} introduce attention to linear model classification. We are planning to express this technique to more complex network including LSTM. 

Both bag of word and word embedding may be tried in experiment. Also, MLP, CNN, RNN, LSTM and other network structures will be tested.

% Representation 

% word - > vector :

% bag of word
% word embedding 

% word thing -> big vector : 

% Triditional : 
%   mean
%   rnn / lstm
%   bi-lstm 

%   LEAM

% text hidden vector text

% hidden vector - > classification

% VAE: 

%   ZSDM : Zero shot document model 

%   Conditional VAE DM

  
% text -> vector

% word embedding , 

% latent variable, 

\section{Experiment Expectation}

\subsection{Dataset}

We are planning to use one of these public dataset, 

\begin{itemize}
  \item  Reuters Newswire Topic Classification (Reuters-21578). A collection of news documents that appeared on Reuters in 1987 indexed by categories. It has 90 classes, 7769 training documents and 3019 testing documents. 
  \item The Extreme Classification Repository: Multi-label Datasets \& Code.
  \item  IMDB Movie Review Sentiment Classification (stanford). A collection of movie reviews from the website imdb.com and their positive or negative sentiment.
  \item News Group Movie Review Sentiment Classification (cornell). A collection of movie reviews from the website imdb.com and their positive or negative sentiment.
\end{itemize}

First two are multi-label dataset. The rest are single-label dataset.

\subsection{Results}

We will compare accuracy for multiple methods. AUC curve may provided.

\section{Appendix}

\subsection{Collaboration}

Everything is a plann and not decided yet.

\begin{itemize}
  \item  Yifei Xu : Writing proposal, implement core code.
  \item  Yaxuan Zhu : Writing report, implement add-on, optimizing code.
  \item Ruiqi Gao : Presentation, Dataset preprocessing, model  training.
\end{itemize}

\subsection{Timetable}

Everything is a plann and not decided yet.

\begin{itemize}
  \item Apr.19th : Finish Proposal
  \item May 15th : Finish core part 
  \item May 30th : Finish alpha version code
  \item Jun.7th : Finalize code, finish report and presentation 
\end{itemize}


\section*{Acknowledgments}

This template is NIPS2017 template downloaded from NIPS2017 website.

This proposal is for 2019 Spring CS269 Project. NO DISTRIBUTION.

\bibliographystyle{plain}
\bibliography{bbib}

\end{document}
